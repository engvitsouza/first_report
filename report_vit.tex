\documentclass[conference,compsoc]{IEEEtran}
\usepackage[utf8]{inputenc}
\usepackage[T1]{fontenc}
\usepackage{lineno,hyperref}
\usepackage{amssymb}
\usepackage[inline]{enumitem}
\usepackage[linesnumbered,ruled,vlined]{algorithm2e}
\usepackage{amsmath}
\usepackage{tabularx}
\usepackage{multirow}
\usepackage{float}
\usepackage{graphicx}
\usepackage{filecontents}
\usepackage{cite}

\begin{document}
\title{An evolutionary compression algorithm for LTE  downlink signals }
\author{\IEEEauthorblockN{Vit\'{o}ria Alencar de Souza}
\IEEEauthorblockA{Department of electrical engineering
Federal University of Par\'{a}\\
Bel\'{e}m - Brazil \\
Email: vitoria.souza@itec.ufpa.br}
}
\IEEEpeerreviewmaketitle
\maketitle

\begin{abstract}
  As the communications has improved the human behavior also changed and it is also 
  creating  new demands and challenges for the next generation of broadband access. The greatest  
  challenges are provide high connection rates and also  concerns  about environmental  
issues. The new architecture of access networks CRAN (centralized radio access network) is 
proposed in order of provide higher data rates and enables the green communications.In CRAN the 
improvements are  due to its centralization but it also creates some impairments such as limitations 
in fronthaul capacity. Thus this article aims improve the compression of baseband signals in a 
fronthaul of CRAN by proposing a new clustering evolutive  method.
\end{abstract}


\section{Introduction}
   
   
  It is projected that the 5G network will be required to deliver data rate about 100 to 1000 
  times the current 4G technology, utilizing radical increase in wireless bandwidths at very 
  high frequencies, extreme cellular network densitification, and massive number of 
  antennas ~\cite{oquesera}.


   Traditional base stations (BSs) comprise either a co-located
   baseband unit (BBU) with a radio unit or a distributed BBU with a remote radio unit  
   (RRU) connected via fiber. For either case, a separate equipment room with supporting facilities 
   such as air conditioning  is required in order for BS deployment, rising the operating expense 
   (OPEX) of the network. Nevertheless since the operating frequency of LTE is    usually higher 
   than that of 2G and 3G, the coverage of an LTE cell is smaller than that of a 2G or 3G cell. As 
   a result, more LTE cells are needed to cover the same area, meaning that more equipment rooms 
   are required and more capital expenditure (CAPEX) is   needed ~\cite{quek2017cloud}.
 

   However according of climate change, rising fossil fuel prices and energy security increase,
   companies and governments around the world are committing great efforts to develop new 
   technologies for the green strategies addressing climate chance globally and facilitating low 
   greenhouse gas (GHG)  development ~\cite{yu2012green}.
   %Currently, the GHG emissions produced by the Information and 
   %Communication Technology (ICT) industry alone are said to be %equivalent to the GHG emissions 
   % of 
   %the entire aviation industry ~\cite{yu2012green}.

    CRAN (centralized radio access network) is new architecture for radio access network who was 
   proposed for the  next generation of cellular networks, the ``C" could means centralized but 
   also collaborative or cloud radio access network.

   In this architecture the baseband units (BBU) and the remote radio heads (RRH) are physically 
   separated, and digitized baseband complex  samples are transported between RRH and BBU 
    ~\cite{cran_arch}.
  
   The  centralization enables many advantages, first the number of equipment rooms 
   for BS placement is greatly reduced, leading to CAPEX reduction. Furthermore, the facilities, 
   especially the air conditioning, could be shared by BBUs in the same central office, leading to 
reduction in a OPEX of the network ~\cite{quek2017cloud}. Then CRAN is also concerned with 
environmental issues.
  
  




\section{Compression principles for LTE signals }
In a LTE downlink signals there is many redundancies  whose are inserted in order of transmit this 
signals over a wireless channel whose impose some distortions. Thus LTE signals has 
some redundancies  in a time and frequency domain that could be removed.

Compression of downlink LTE signals is a process that is performed before fronthaul, it aims
remove redundancies of a baseband LTE signal. 


The fundamental idea of this process  is remove  all the 
redundancies in a compression module whose is inserted before the fronthaul and so that they can be 
reinserted after the fronthaul in  a decompression module. Then  the fronthaul capacity is improved.


Compression  process  and the proposed improvements are described in the next sections.

\subsection{Remove time redundancies}
\subsection{Remove frequency  redundancies}

\subsection{Clustering samples }



\section{Proposed method}


\bibliographystyle{./IEEEtran}
\bibliography{./IEEEabrv,./telecom.bib,./dsp.bib}
\end{document}